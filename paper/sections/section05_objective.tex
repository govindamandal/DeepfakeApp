\section{Research Objective and Approach}
The objective of this research is to enhance the effectiveness of deepfake video detection using Vision Transformers (ViTs). Deepfakes pose significant societal risks, and while Convolutional Neural Networks (CNNs) have been widely employed in video forgery detection, they often struggle with the generalization of spatiotemporal information inherent in videos. To address this challenge, the study leverages the transformer architecture with its inherent ability to capture long-range dependencies and spatial relationships between frames, aiming to improve detection accuracy and robustness.
\subsection{The proposed approach involves:}
\begin{itemize}
    \item \textbf{Utilizing Vision Transformers (ViT)}:
        Instead of relying solely on CNNs, the ViT is used for better feature extraction from video frames. This transformer-based architecture captures both spatial and temporal patterns from the frames.
    \item \textbf{Dataset Enhancement}:
        To prevent over-fitting, a large dataset of deepfake videos is employed (FaceForensics++), ensuring a diverse range of manipulations and challenges for the model.
    \item \textbf{Training and Optimization}:
        The model is trained using advanced optimization techniques to ensure high accuracy, precision, and generalizability across different types of deepfakes.
\end{itemize}
