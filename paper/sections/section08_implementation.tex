\textbf{Deepfakes} represent manipulated media—be it images, videos, or audio—that alter the original context, often creating misleading content that can deceive viewers. Using advanced AI techniques such as Generative Adversarial Networks (GANs), deepfake creators can modify facial expressions, voices, or even simulate entire videos. Notably, these manipulations can be convincing, as demonstrated by viral videos, including fabricated celebrity appearances.\\


 While deepfakes may serve benign purposes such as satire or entertainment, they also pose significant threats across various domains. From privacy concerns to political manipulation and corporate espionage, deepfakes have the potential to undermine public trust and damage reputations. The rapid improvement in deepfake quality requires equally advanced detection mechanisms to mitigate their spread.\\

Traditional methods for detecting deepfakes rely heavily on convolutional neural networks (CNNs) and recurrent neural networks (RNNs), which focus on local pixel patterns. However, these models may struggle with identifying subtle or widespread inconsistencies across video frames. In contrast, Vision Transformers (ViTs) excel at capturing long-range dependencies by using a self-attention mechanism, making them a promising approach for detecting deepfakes. By processing entire video frames as sequences, ViTs can identify minor artifacts and manipulations more effectively.\\

In this paper, we present a novel deepfake detection system using ViTs, which outperforms conventional CNN-based approaches. Our work contributes to improving the detection accuracy and robustness of models against adversarial manipulations. The following sections detail our methodology, experiments, and results, underscoring the potential of ViTs in enhancing digital content authenticity.\\